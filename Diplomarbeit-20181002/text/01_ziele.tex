\section{Hauptziele}
\begin{description}
\item Ziel-H 1      Wochenkochplan\\
Ein Wochenkochplan ist für die laufende Woche ausgewertet und übersichtlich dargestellt. Für jeden Tag der Woche sind min. 0 und max. 4 Mahlzeiten vorhanden. Das kann in den Präferenzen eingestellt werden. Ebenso kann angegeben werden welche Mahlzeiten man an welchem Tag haben möchte. Für jede Mahlzeit ist eine Zutatenliste und ein Rezept vorhanden.  

\item Ziel-H 2      Tags\\
Die Tags der Zutaten umfassen Mikronährstoffe, Allergene und Besonderheiten des Produkts. Die Tags der Speisen umfassen die einzelnen Zutaten. Mit allen Tags der Zutaten kann man auswerten welche Eigenschaften eine Mahlzeit hat (z.B. Unverträglichkeiten) um somit den Plan nach den Userpräferenzen einstellen zu können. 

\item Ziel-H 3      Einkaufsliste\\
Anhand des Speiseplans für die ganze Woche kann eine Einkaufsliste mit allen Zutaten erstellt werden. Zutaten sind in der lokalen Datenbank gespeichert und können editiert, hinzugefügt und gelöscht werden.  

\item Ziel-H 4      Userpräferenzen\\
Im Usermenü kann mit Hilfe eines Formulars eingestellt werden wie viele und welche Art von Mahlzeiten (Frühstück, Mittag, Abendessen, Snack) man an einem bestimmten Wochentag haben möchte. Ebenso können individuelle Essensgewohnheiten (z.B. dreimal die Woche Fleisch, jeden Tag eine Suppe), Allergien und ungewollte Speisen eingetragen werden. All das wird mit Hilfe von Tags, die die einzelnen Speisen und Zutaten zugeordnet haben, ausgewertet.


\end{description}
\section{Optionale Ziele}
\begin{description}
\item Ziel-O 1      Effizienz\\
Um die Download- und Auswertungszeiten zu verkürzen, werden Daten für bis zu 5 Wochen aus der API zwischengespeichert und in der lokalen Datenbank abgelegt. Die neu geladenen Rezepte oder Zutaten werden mit den schon vorhandenen verglichen, dies spart Zeit. 

\item Ziel-O 2      Gänge Menü\\
Man kann sich für gewünschte Tage (z.B. Feiertage) ein bis zu 5 Gänge Menü generieren lassen.

\item Ziel-O 3      Erweiterte Userpräferenzen\\
User können anhand einer Checkliste in dem Formular mit Hilfe von Speisen- und Zutatentags den Speiseplan nach bestimmten Ernährungskonzepten einstellen. Zum Beispiel: „high-protein“, „laktosefrei“, „vegetarisch“ usw.

\item Ziel-O 4      Offline\\
Der Wochenplan und die Einkaufsliste lassen sich exportieren und sind offline abrufbar. So kann man zum Beispiel im Geschäft, sollte man kein Internetzugang haben, offline die Produktliste und den Plan ansehen.

\end{description}
\section{Nicht-Ziele}
\begin{description}
\item Ziel-N 1      Android-/IOS-App\\
Die Applikation ist als native Android- oder IOS-App verfügbar.

\item Ziel-N 2      Rezeptdatenbank\\
Das Projektteam erstellt eine eigene Rezeptdatenbank.

\end{description}
