\hypertarget{kapitel-aus-der-anderen-datei}{%
\section{Kapitel aus der anderen
Datei}\label{kapitel-aus-der-anderen-datei}}

Dieses Kapitel wurde als \emph{diplomarbeit2.md} geschrieben und dann
mit \emph{pandoc} in \TeX~umgewandelt.

\begin{lstlisting}[language=bash]
pandoc --listings -s diplomarbeit2.md -o diplomarbeit2.tex 
\end{lstlisting}

Wie man sieht ist das ganz einfach, sogar Listings sind möglich. Weitere
Optionen sind möglich bzw. sinnvoll -- siehe Kapitel \ref{skripts},
Seite \pageref{skripts}. Man beachte die von Pandoc automatisch
generierten Label für Querverweise.

Vorschlag zur Durchführung:

\begin{itemize}
\tightlist
\item
  ein Ordner mit den Pandoc-Dateien
\item
  ein Skript/Batch-Datei erzeugt daraus die Latex-Dateien

  \begin{itemize}
  \tightlist
  \item
    in einem neuen Ordner -- das erhöht die Übersichtlichkeit
  \end{itemize}
\item
  die Latex-Dateien werden dann in das Hauptdokument eingebunden
\end{itemize}

Und nun zu einem Bild. Floating Elemente können auch auf der nächsten
Seite sein, ein Verweis auf das Bild ist deshalb sinnvoll: siehe Bild
\ref{Bild1} oder siehe \abb{Bild1}.

\begin{figure}
\centering
\includegraphics{HTL3RLogo.png}
\caption{Der Text steht unterhalb\label{Bild1}}
\end{figure}

Man kann auch die Breite aber auch angeben.

\begin{figure}
\centering
\includegraphics[width=12cm,height=\textheight]{HTL3RLogo.png}
\caption{Das kleinere Bild}
\end{figure}

Oder ganz klein (Bild \ref{Bild3}).

\begin{figure}
\centering
\includegraphics[width=2cm,height=\textheight]{HTL3RLogo.png}
\caption{Das ganz kleine Bild\label{Bild3}}
\end{figure}

Auch Listen sind kein Problem, wichtig sind nur Leerzeilen zwischen den
Listenpunkten. Hier sieht man eine einfache Aufzählung.

\begin{itemize}
\item
  wichtig
\item
  auch ganz lange Texte können bei Listen geschrieben werden.

  Sogar mehrere Absätze sind möglich.
\item
  Ende der Liste.
\end{itemize}

Welches Zeichen am Anfang der Liste steht ist dabei leicht einzustellen,
im \emph{pandoc} Manual gibt es nähere Infos:

\begin{enumerate}
\def\labelenumi{\arabic{enumi}.}
\item
  eins
\item
  zwei

  \begin{enumerate}
  \def\labelenumii{\roman{enumii}.}
  \tightlist
  \item
    zwei eins -- Mindestens 4 Zeichen eingerückt
  \item
    zwei zwei
  \end{enumerate}
\item
  drei. \emph{Pandoc} zählt richtig, das Zeichen am Anfang der Zeile ist
  nur ein Muster!
\end{enumerate}

Mit den richtigen Optionen werden URLs automatisch richtig dargestellt
und sind im fertigen Pdf auch \emph{klickbar}:
\url{http://pandoc.org/README.html}
